\chapter{PION NUCLEON SCATTERING}



\section{History of Pion Nucleon Scattering}

Studying pion nucleon scattering is crucial for advancing our understanding of subatomic particles and their interactions. Pion nucleon scattering experiments were started several years after the discovery of pion.

Hideki Yukawa hypothesized about the existence of pions in 1935 \cite{Yukawa:1935xg}. During that period, our understanding of subatomic particles was very limited. The electron, proton, and neutron were discovered, and they were known as subatomic particles. Their properties, such as mass, electric charge, spin, and size, were known. Yakuwa proposed the strong nuclear force because the electromagnetic force was unable to keep nucleons (protons and neutrons) stably inside the nucleus. He also proposed the existence of a new particle, the pion, which is responsible for mediating this new force. His idea was that pions are exchanged between protons and neutrons to hold them together, and he boldly predicted the mass of this unseen pion to be around 100 MeV. He deduced this value by assuming the effective range of this force to be roughly the size of a nucleon. Even though his hypothesis is not very accurate to the level of today's understanding, it was a very good proximation that led Yakuwa to the 1949 Nobel Prize in physics.

Then scientists started looking for a new particle with a mass of roughly 100 MeV. High energy cosmic rays coming from space can interact with the atmosphere and produce new subatomic particles. Therefore, particle detectors were placed at high altitudes in the mountains. First, the muon was detected, and it was confused with the Yukawa's particle, the pion. Later in 1947, Cecil Powell and collaborators discovered the first charged pion \cite{Lattes1947ProcessesIC} on a photographic emulsion plate. By examining the coulomb scattering-related changes in the path and counting the grains on the plate, it was possible to determine how heavy these pions were. They reported masses of 100 MeV to 150 MeV for discovered pions \cite{CFPowell_1950}. For his work on this finding, Powell was awarded the Nobel Prize in 1950.

A year later, pions were artificially produced at the Berkeley Cyclotron, California\cite{FirstPionsBerkeley}. This was done by bombarding $1/16$ inch carbon target with an alpha particle beam of energy 380 MeV. Since the pion is a very unstable particle, these artificially produced pions also contained their decay products, muons and electrons. They reported observing pions with a mass of around 150 MeV and kinetic energy between 2 MeV and 5 MeV. Ever since, physicists have begun to produce pion beams with higher energies and intensities for use them in new experiments.


In the early 1950s, pion nucleon scattering experiments were started. Enrico Fermi and collaborators published the first high-quality pion nucleon data set. They conducted a negative pion and proton scattering experiment at the Chicago cyclotron and published a total cross section data at bunch of different energy values between 80 Mev and 220 Mev  \cite{PiNScatteringFermi1952}. Since then, pion nucleon scattering experiments have been conducted worldwide. A significant amount of good-quality data was produced at the so-called Meson Factories: PSI (Paul Scherrer Institute, Switzerland), TRIUMF (Canada's particle accelerator center), and LAMPF (Los Alamos Meson Physics Facility).


\section{Introduction to QCD}

\subsection{Symmetries of QCD}


According to the Standard Model, the theory that describes the Strong Nuclear Force is known as Quantum Chromodynamics (QCD). Strong interaction is described by a local, non-Abelian $SU(3)_C$ gauge theory. It describes the extremely short-ranged ‘color force’ that binds quarks together and the gluons that mediate the force. Quarks and gluons are asymptotically free at short distances and they are confined at large distances: only colorless bound states (hadrons) are observed in experiments, and no quark or gluon has ever been directly observed.

\subsection{Asymptotic Freedom and Confinement}

\subsection{Effective Field Theories}
